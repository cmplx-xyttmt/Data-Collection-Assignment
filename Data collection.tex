\documentclass[11pt]{article} 

\usepackage[utf8]{inputenc}
\usepackage{geometry}
\geometry{a4paper} 

\title{Calculating accurate path costs by tracking traffic.}
\author{Owomugisha Isaac \\ 15/U/12351/PS \\ 215004131}

\begin{document}
\maketitle

\section{Introduction}
This concept paper highlights a method for calculating how long it takes to drive from place in a city to another. The method takes into account factors such as distance between source and destination and traffic congestion at a given place.\\
The path cost between two points as used in this paper is a measure of how long it takes one to get from one point the other. This cost can be calculated by actually driving between the 2 points many times using different routes, however, that is inefficient and expensive. This paper highlights a method in which data is collected in order to calculate these paths easily.

\section{Problem Statement}
Given two points, and a path that leads from one point to the other, calculate the cost for that path. \\
If there are many paths possible, this calculation would help in choosing the most time-efficient path to take.

\section{Objectives}
\begin{itemize}
\item To collect the image and geo data necessary.
\item To use the data collected to calculate how long it will take to get from one place to another.
\end{itemize}

\section{Methodology}
Using ODK, "point" data will be collected. Point data consists of 4 fields:point in journey, traffic picture, time, and location.
\begin{itemize}
\item Point in journey: This is an identifier for where you are in the journey. Point 0 indicates the starting point. Subsequent points are $1, 2,3, \cdots $

\item Traffic picture: At each point, a picture of the road is taken. The point of this picture is to determine how congested that particular road is. An overhead picture would be prefferable since it can have a better view and coverage of the road.

\item Time: This is the time at which this data point is being recorded. This will help in determining the actual time it took to travel between data points.

\item Location: This consists of the coordinates of the place where this data point is being collected. This information will be used to calculate distances between points. 
\end{itemize}
\end{document}
